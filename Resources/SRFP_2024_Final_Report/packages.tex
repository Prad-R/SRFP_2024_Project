%----------------------------------------%
% Here are where all the packages that you
% wish to use are stored, these are just
% a few necessary for basic structure.
%----------------------------------------%
\usepackage{amsmath}
\usepackage{lipsum}  %% Included for testing purposes
\usepackage{tikz}
\usepackage{algorithm2e}
\usepackage{microtype}
\usepackage{fancyhdr}
\usepackage{datetime}
\usepackage{mathptmx}
\usepackage{setspace}
\usepackage[margin=2.5cm]{geometry}
\usepackage[backend=biber]{biblatex}

\usepackage{mathtools,amsthm} % Enable useful mathematical symbols/environments
\usepackage{graphicx} % Enable graphics
\usepackage{titlesec,microtype} % enable various formatting commands
\usepackage[T1]{fontenc}

  
  \usepackage{xcolor} % Enable coloured elements
  \definecolor{mypurple}{HTML}{622567} %%% Purple
  \definecolor{myred}{HTML}{D55C19} %%%EssexOrange
  \definecolor{myblue}{HTML}{007A87} %%%Seagrass
  
  % For technical reasons, hyperref should be loaded after all other packages
  \usepackage[colorlinks,linkcolor=myblue,citecolor=mypurple]{hyperref}
  
  \renewcommand{\baselinestretch}{1.5} % 1.5 line spacing
  
  % Define \begin{theorem}, \end{theorem}, etc.
  \theoremstyle{plain} % The following will be italicised
  \newtheorem{theorem}{Theorem}[chapter]
  \newtheorem{lemma}[theorem]{Lemma}
  \newtheorem{proposition}[theorem]{Proposition}
  \newtheorem{corollary}[theorem]{Corollary}
  
  \theoremstyle{definition} % The following environments will not use italics
  \newtheorem{definition}[theorem]{Definition}
  \newtheorem{example}[theorem]{Example}
  
  \theoremstyle{remark} % The following environments will not use italics or bold titles
  \newtheorem{remark}[theorem]{Remark}
  
  \numberwithin{equation}{chapter}
  
  % Fancy headings
  \pagestyle{fancy}
  \setlength{\headheight}{15pt}
  \fancyheadoffset[LE,RO]{0pt}
  \renewcommand{\chaptermark}[1]{\markboth{#1}{}}
  \renewcommand{\sectionmark}[1]{\markright{\thesection\ #1}}
  \fancyhf{}
  \fancyhead[LE]{\makebox[0pt][l]{\thepage}\hfill\leftmark}
  \fancyhead[RO]{\rightmark\hfill\makebox[0pt][r]{\thepage}}
  \fancypagestyle{plain}{%
  	\fancyhead{} % get rid of headers
  	\renewcommand{\headrulewidth}{0pt} % and the line
  }
  
  % Fancy chapter numbers
  \titleformat{\chapter}[display]
  {\normalfont\bfseries\color{myred}}
  {\filleft\hspace*{-60pt}%
  	\rotatebox[origin=c]{90}{%
  		\normalfont\color{black}\Large%
  		\textls[180]{\textsc{\chaptertitlename}}%
  	}
  	\hspace{10pt}%
  	{\setlength\fboxsep{0pt}%
  		\colorbox{myred}{\parbox[c][3cm][c]{2.5cm}{%
  				\centering\color{white}\fontsize{80}{90}\normalfont\thechapter}%
  		}
  	}
  }
  {10pt}
  {\titlerule[2.5pt]\vskip3pt\titlerule\vskip4pt\LARGE\normalfont}

