\newpage

\chapter{Conclusions}

We started by understanding the underlying assumptions of continuum fluid mechanics, and the conditions under which these assumptions start to fail. Progressing from there, we looked at alternative methods to model fluid flow in these regimes where continuum assumptions fail, exploring the mathematical tools on a broad level. After gaining a preliminary understanding of the underlying mathematics and physics, we discuss the Direct Simulation Monte Carlo (DSMC) algorithm to simulate such flows. Consequently, we moved on to introduce an open-source tool called SPARTA and explored some existing studies that verify its validity by comparing it with accepted findings. \\

\no We then moved on to utilizing SPARTA for our simulations, starting with 1D Fourier Flow where we solved the system for 1D heat transfer under various conditions, exploring deviations and providing the reasons for the same. After, this we moved on to 2D simulations of flow around a cylinder and explored ray tracing tools to visualize the flow. We also plotted the lift and drag variations to see if we indeed got the lift and drag values that we expected. \\

\no In the end, we discussed some future directions that one can explore with SPARTA and pointed out an interesting problem involving the Martian atmosphere. Some relevant data and their visualizations were also provided and discussed.