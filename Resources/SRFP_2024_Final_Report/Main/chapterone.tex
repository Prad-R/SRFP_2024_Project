\newpage
\chapter{Boltzmann Equation}

The Boltzmann equation describes the statistical behaviour of a thermodynamic system not in a state of equilibrium. The equation is not the result of analysing every individual particle but is rather a result of considering a probability distribution of the position and momentum of a typical particle. In other words, it's the probability that a particle occupies a very small region of space centred around $ \vec{r} $ (represented as $ d^3\vec{r} $) and momentum nearly equal to $ \vec{p} $ (represented as $ d^3\vec{p} $) at an instant of time.

\section{The Phase Space and the Density Function}

The set of all positions $\vec{r}$ and momenta $\vec{p}$ parameterised by time $t$ is called the \textit{phase space} of the system. This space is 6-dimensional. If we define a probability distribution function $ f(\vec{r}, \vec{p}, t) $, then we obtain the number of particles $(dN)$ in $ d^3 \vec{r} d^3 \vec{p}$ as,

\begin{equation}
	dN = f(\vec{r}, \vec{p}, t) d^3 \vec{r} d^3 \vec{p}
\end{equation}

\no Integrating over positions and momenta, we obtain,

\begin{equation}
	N = \int_{momenta} d^3 \vec{p} \int_{positions} d^3 \vec{r} f(\vec{r}, \vec{p}, t)
\end{equation}

\section{Principal Statement}

\no The general equation is given by,

\begin{equation}
	\frac{df}{dt} = \left( \frac{\partial f}{\partial t} \right)_{force} + \left( \frac{\partial f}{\partial t} \right)_{diff} + \left( \frac{\partial f}{\partial t} \right)_{coll}
\end{equation}

\no 	where the \textit{force} term corresponds to the forces exerted on the particles by an external influence, \textit{diff} refers to the diffusion of the particles and \textit{coll} refers to the forces due to particle collisions.

\subsection{Force and Diffusion Terms}

Suppose there are particles with position $\vec{r} $ in $ d^3 \vec{r}$ and momentum $ \vec{p} $ in $ d^3 \vec{p}$ at time  $t$. If a force $\vec{F}$ acts instantaneously on all particles, then at $ t + \Delta t $, we have $ \vec{r} + \Delta \vec{r} = \vec{r} + \frac{\vec{p}}{m} \Delta t $ and $ \vec{p} + \Delta \vec{p} = \vec{p} + \vec{F} \Delta t $. In the absence of collisions, if we write an equation for the number of particles, we obtain,

\begin{equation}
	f(\vec{r} + \frac{\vec{p}}{m} \Delta t, \vec{p} + \vec{F} \Delta t, t + \Delta t) d^3\vec{r} d^3\vec{p} = f(\vec{r}, \vec{p}, t) d^3\vec{r} d^3\vec{p}
\end{equation}

\no In the above equation, we have used the fact that the phase space volume element is constant. However, if there are collisions, the particle density in phase space volume changes, and as a result we obtain,

\begin{align}
		dN_{coll} &= \left( \frac{\partial f}{\partial t} \right)_{coll} \Delta t d^3\vec{r} d^3\vec{p} \\	
		&=\left( f(\vec{r} + \frac{\vec{p}}{m} \Delta t, \vec{p} + \vec{F} \Delta t, t + \Delta t) d^3\vec{r} d^3\vec{p} - f(\vec{r}, \vec{p}, t) \right)d^3\vec{r} d^3\vec{p} \\
		&= \Delta f d^3\vec{r} d^3\vec{p}
\end{align}

\no Dividing by the phase space volume and $ \Delta t $ and taking $ \Delta f, \Delta t \to 0 $, we obtain,

\begin{equation}
	\frac{df}{dt} = \left( \frac{\partial f}{\partial t} \right)_{coll}
\end{equation}

\no The total differential of $ f $ is,

\begin{align}
	df &= \frac{\partial f}{\partial t} dt + \left( \frac{\partial f}{\partial x} dx + \frac{\partial f}{\partial y} dy + \frac{\partial f}{\partial z} dz \right) + \left( \frac{\partial f}{\partial p_x} dp_x + \frac{\partial f}{\partial p_y} dp_y + \frac{\partial f}{\partial p_z} dp_z \right) \\
	&= \frac{\partial f}{\partial t} dt + \vec{\nabla}f \cdot \frac{\vec{p}}{m}dt + \vec{\nabla}_p f \cdot \vec{F}dt
\end{align}

\no Dividing the above equation by $dt$, we obtain the final statement given as,

\begin{equation}
	\frac{\partial f}{\partial t}+ \vec{\nabla}f \cdot \frac{\vec{p}}{m} + \vec{\nabla}_p f \cdot \vec{F} = \left( \frac{\partial f}{\partial t} \right)_{coll}
\end{equation}

\section{Conservation Equations}

From the Boltzmann equation, it is possible to obtain the conservation equations identical to the ones described by fluid mechanics by taking $n^{th}$ moments of the equation with the velocity field. The number density is given by,

\begin{equation}
	n = \int_{momenta}	f d^3\vec{p}
\end{equation}

\no and the average value of any function $ A $ is given by,

\begin{equation}
	\langle A \rangle = \frac{1}{n} \int_{momenta} A f d^3\vec{p}
\end{equation}

\no If we multiply the entire Boltzmann equation with $A$ and integrate it with respect to momentum volume, we obtain 4 equations given as,

\begin{equation}
	\int A \frac{\partial f}{\partial t} d^3\vec{p} = \frac{\partial}{\partial t} \left( n \langle A \rangle \right)
\end{equation}

\begin{equation}
	\int \frac{p_jA}{m} \frac{\partial f}{\partial x_j} d^3\vec{p} = \frac{1}{m} \frac{\partial}{\partial x_j} \left( n \left\langle Ap_j \right\rangle \right)
\end{equation}

\begin{equation}
	\int A F_j \frac{\partial f}{\partial p_j} d^3\vec{p} = - n F_j \left\langle \frac{\partial A}{\partial p_j} \right\rangle
\end{equation}

\begin{equation}
	\int A \left( \frac{\partial f}{\partial t} \right)_{coll} d^3 \vec{p} = 0
\end{equation}

\no If we take the zeroth moment of these equations \textit{i.e.,} $A = m(v_i)^0 = m$ we obtain the conservation of mass equation. \\

\no If we take the first moment of these equations \textit{i.e.,} $A = m(v_i)^1 = mv_i$ we obtain the conservation of momentum equation. \\

\no If we take the second moment of these equations \textit{i.e.,} $A = m(v_i)^2 = mv_i^2$ we obtain the conservation of energy equation.

\section{Limitations of the Boltzmann Equation}

The primary limitation of the Boltzmann equation is that it assumes that particles are point-sized \textit{i.e.,} they have 0 size. A modification of the Boltzmann equation exists, called the \textbf{Enskog equation} and this takes into account the finite size of the particles. Furthermore, no degrees of freedom apart from translation is assumed for particles by the Boltzmann equation. Additionally, many fluids have not only binary but also ternary and higher-order collisions that the Boltzmann equation simply doesn't capture.